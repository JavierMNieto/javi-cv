\documentclass[margin,line]{res}

% For links
\usepackage{hyperref}
\hypersetup{colorlinks=True, urlcolor=cyan}

% Suppress bibliography numbering
\makeatletter
\renewcommand\@biblabel[1]{}
\makeatother

% Macro for auto-bolding author name in bibliography
\usepackage{xstring}
\def\FormatName#1{ \IfSubStr{#1}{Barnowski}{\textbf{#1}}{#1} }

% For recovering section titles after suppressing them for bibtex
\let\resumesection\section

% Multiple bibliographies
\usepackage{multibib}
\newcites{publications}{Publications}
\newcites{presentations}{Presentations}
\newcites{posters}{Posters}

\oddsidemargin -.5in
\evensidemargin -.5in
\textwidth=6.0in
\itemsep=0in
\parsep=0in

\newenvironment{list1}{
  \begin{list}{\ding{113}}{%
      \setlength{\itemsep}{0in}
      \setlength{\parsep}{0in} \setlength{\parskip}{0in}
      \setlength{\topsep}{0in} \setlength{\partopsep}{0in} 
      \setlength{\leftmargin}{0.17in}}}{\end{list}}
\newenvironment{list2}{
  \begin{list}{$\bullet$}{%
      \setlength{\itemsep}{0in}
      \setlength{\parsep}{0in} \setlength{\parskip}{0in}
      \setlength{\topsep}{0in} \setlength{\partopsep}{0in} 
      \setlength{\leftmargin}{0.2in}}}{\end{list}}


\begin{document}

\name{Ross W. Barnowski, Ph.D \vspace*{.1in}}
% HACK: res.cls has listing for addresses - use instead to show current 
% position
\address{\large Senior Research Software Engineer}
%\address{\large Department of Nuclear Engineering}
%\address{\large Berkeley Institute for Data Science, UC Berkeley}
\address{\large Biology and Biological Engineering, Caltech}

\begin{resume}
\section{\sc Contact Information}
\vspace{.05in}
\begin{tabular}{@{}p{3in}p{4in}}
California Institute of Technology & {\bf Phone:}   (847)224-5917 \\
1200 E California Blvd. MC 114-96  & {\bf E-mail:}  {\tt rossbar@caltech.edu} \\
Pasadena CA, 91125                 & {\bf GitHub:} \url{https://github.com/rossbar} \\
\end{tabular}


\section{\sc Research Interests}
open-source software, scientific computing, scientific Python, image reconstruction,
nuclear instrumentation, gamma-ray spectroscopy and imaging, nuclear medicine
(diagnostic imaging, radiotherapy), robotics, sensor fusion, SLAM

\section{\sc Education}
{\bf University of California, Berkeley} - Berkeley, California \hfill 09/2010 - 06/2016\\
Ph.D, Nuclear Engineering - August 2016 | GPA: 3.98/4.0
\begin{list2}
\item {\bf \small Dissertation:} {\em``Development and Evaluation of Real-Time
                                        Volumetric Compton Gamma-Ray Imaging''} 
\item {\bf \small Advisor:}  Kai M. Vetter
\item {\bf \small Coursework:} Advanced Radiation Instrumentation Laboratory,
                               Radiation Transport Simulation and Modelling,
                               Medical Imaging Signals and Systems, Computer
                               Vision, Applications in Parallel Computing,
                               Python for Scientific Computation
\end{list2}

{\bf University of Michigan} - Ann Arbor, Michigan \hfill 09/2006 - 04/2010\\
B.S.E. Nuclear Engineering and Radiological Science - April 2010 | GPA 3.98/4.0
\begin{list1}
\item[] {\bf \small Coursework:} Nuclear Reactor Design, Radiation Detection 
                                 and Measurement, Nuclear Reactor Kinetics,
                                 General Radiation Laboratory, Nuclear Physics,
                                 Quantum Mechanics, C++ for Sci/Eng
\end{list1}

\section{\sc Experience}
{\bf California Institute of Technology} \\ 
{\em Senior Research Software Engineer | Biology and Biological Engineering (BBE)} \hfill 12/2022 - Present

Research software development, maintenance and systems administration
supporting AI/ML research for spatialomics applications.
In this role, I also directly contribute to the open-source scientific Python
ecosystem which is actively used in the lab's research.

\begin{list1}
  \item[] {\bf \small Research software development and maintenance} Supporting
          various projects across the lab with reproducible, scalable analysis
          of spatialomics data.
  \begin{list2}
    \item Deep learning for single-cell and tissue data, including cell
          segmentation and cell-type determination from multiplexed imaging
          modalities.
  \item Maintenance of the \href{https://deepcell.org}{Deepcell} suite of
          open-source libraries for single-cell analysis.
  \end{list2}
  \item[] {\bf \small NumPy and NetworkX core maintainer}
  \begin{list2}
    \item Active maintainer of NumPy (array computation) and NetworkX
          (graph theory and network analysis) libraries.
    \item Code development, maintenance, and (most importantly) review; community 
          organization, and project infrastructure.
  \end{list2}
  \item[] {\bf \small Infrastructure and systems administration}
  \begin{list2}
    \item Manage distributed, on-prem compute infrastructure for the lab
          comprising $>10$ machines, $>30$ GPUs, and $>300$TB with $>30$ users.
  \end{list2}
\end{list1}

{\bf University of California, Berkeley} \\ 
{\em Scientific Software Developer | Berkeley Institute for Data Science} \hfill 11/2019 - 11/2022

\begin{list1}
  \item[] {\bf \small Scientific Python Ecosystem} Scientific Python Ecosystem
          coordination and development: \url{https://scientific-python.org}
  \begin{list2}
    \item Support development and sceintific computing applications of the
          scientific Python ecosystem, comprising NumPy, SciPy, Matplotlib,
          scikit-learn, scikit-image, NetworkX, etc.
    \item Improve ecosystem coordination and robustness via community
          development, implementing cross-project decision-making processes,
          and general support of ecosystem projects.
  \end{list2}
  \item[] {\bf \small NumPy and NetworkX core maintainer}
  \begin{list2}
    \item Active maintainer of NumPy (array computation) and NetworkX
          (graph theory and network analysis) libraries.
    \item Code development, documentation \& testing, code review, community 
          organization, and project infrastructure.
  \end{list2}
  \item[] {\bf \small Mentorship \& Outreach} 
  \begin{list2}
    \item Mentor for outreach/onboarding programs such as Google season of code
          (NetworkX 3x), Google season of docs (NumPy 2x), and Outreachy
          (NetworkX 1x).
    \item Open source development outreach program for the African Masters in
          Machine Intelligence (AMMI) program from the African Institute for
          Mathematical Sciences (AIMS): 
          \url{https://bids-numpy.github.io/workshop\_site\_AIMS\_2021/}
  \end{list2}
\end{list1}


{\em Assistant Research Scientist\footnotemark | Department of Nuclear Engineering} \hfill 08/2018 - 03/2019
\footnotetext{The period from 08/2018-12/2018 consisted of a joint role 
              comprising a 60\% appointment as assistant research scientist, 
              and a 40\% appointment as lecturer.}

\begin{list1}
  \item[] {\bf \small Project:} Basic Research - Plastic Scintillator for Unmanned Aerial Systems
  \begin{list2}
    \item Preliminary investigation of the incorporation of plastic 
          scintillation material as structural material in unmanned aerial
          systems.
    \item Led acquisition and initial characterization of plastic scintillator
          material, including development of data acquisition and analysis
          pipeline for plastic scintillators capable of pulse-shape
          discrimination for particle identification.
  \end{list2}
  \item[] {\bf \small Student Mentorship}
  \begin{list2}
    \item Graduate student mentorship for the development and maintenance of
          radiation instrumentation systems, including gamma-ray imaging
          devices based on high-purity germanium and inorganic scintillators.
    \item Training for graduate students on cryogenics, vacuum systems,
          and front-end electronics.
  \end{list2}
  \item[] {\bf \small Data Analysis and Technical Support}
  \begin{list2}
    \item Image reconstruction and data analysis in support of projects 
          related to the mapping of nuclear contamination in Fukushima and
          Chernobyl.
    \item Simulation studies in support of gamma-ray imaging for
          material studies related to uranium enrichment and storage.
    \item Continued firmware/software development and integration support for
          the PRISM instrument (see LBNL postdoc description).
  \end{list2}
\end{list1}

{\em Lecturer | Department of Nuclear Engineering} \hfill 08/2018 - 12/2018

\begin{list1}
  \item[] {\bf \small Course:} \href{https://ne204-fall2018.github.io}{NE 204: Advanced Concepts in Radiation
Detection and Measurement}
  \begin{list2}
    \item Graduate-level, 3 credit-hour laboratory course on radiation
          instrumentation and measurement.
    \item Lecture component emphasizing digital signal processing, physics and
          operation of semiconductor and scintillation-based detectors, and
          spectroscopic and imaging applications of radiation instrumentation.
    \item Laboratory component emphasizing digital signal processing for
          radiation spectroscopy and timing applications, and multi-channel
          radiation detectors for neutron and gamma-ray detection, spectroscopy,
          and imaging.
    \item Computational component emphasizing reproducibile and collaborative
          workflows with git, \LaTeX, and the scientific python ecosystem. 
  \end{list2}
\end{list1}

{\bf Lawrence Berkeley National Laboratory}, Berkeley, CA \\
{\em Postdoctoral Researcher | Applied Nuclear Physics} \hfill 09/2016 - 07/2018

\begin{list1}
  \item[] {\bf \small Project:} Portable Radiation Imaging Spectroscopy and Mapping (PRISM)
  \begin{list2}
    \item Development of software for multichannel data acquisition system.
          Firmware integration and testing; network communication for control
          and data I/O.
          Full system integration including front-end design for HW/FW
          developers (engineering interface) and users.
          Device applications in source search and gamma-ray mapping | 
          enhanced imaging and localization capabilities.
    \item Development and implementation of gamma-ray imaging modalities
          including Compton and proximity-based localization algorithms.
          Direct application to nuclear contamination remediation in 
          Fukushima, Japan.
  \end{list2}
  \item[] {\bf \small Project:} Small-Animal Molecular Imaging
  \begin{list2}
    \item Near-field, high-resolution Compton tomography for molecular imaging
          with small-animal models.
          Development and assembly of tomographic system as well as data 
          analysis and image reconstruction algoritms.
          Proposed applications for radionuclide uptake studies and theranostics
          for radiophamaceutical development.
  \end{list2}
\end{list1}

{\em Graduate Student Researcher | Applied Nuclear Physics} \hfill 06/2011 - 05/2016

\begin{list1}
  \item[] {\bf \small Project:} Volumetric Gamma-Ray Imaging
  \begin{list2}
    \item Develop and evaluate real-time 3D gamma-ray imaging techniques based
          on sensor fusion and real-time SLAM algorithms.
    \item Approach successfully demonstrated with portable high-purity 
          germanium imager as well as hand-portable imager based on 
          room-temperature semiconductor gamma-ray detectors.
    \item Software contributions include multithreaded application framework
          written in Python/C for interfacing with commercial DAQ hardware and
          implementing real-time gamma-ray imaging analysis.
  \end{list2}
\end{list1}

{\bf Lawrence Livermore National Laboratory}, Livermore, CA \\ 
{\em Undergraduate Researcher | Nuclear Data Group} \hfill 06/2009 - 08/2009 \& 06/2010 - 07/2010

\begin{list1}
  \item[] {\bf \small Project:} Benchmarking of Evaluated Nuclear Data Library
                                (ENDL) with Monte Carlo \& deterministic
                                particle transport codes
  \begin{list2}
    \item Generate simulations using Monte Carlo \& deterministic code packages
          for benchmarking nuclear cross-section data in the (ENDL) libraries.
    \item Scripting for automation of simulation input and data analysis.
          Experience with SLURM for job submission on cluster systems.
    \item Listed as co-author on release of ENDL2011.
  \end{list2}
\end{list1}

{\bf Argonne National Laboratory}, Argonne, IL \\ 
{\em SULI Summer Intern} \hfill 05/2008 - 08/2008

\begin{list1}
  \item[] {\bf \small Project:} Millimeter-wave radar for remote detection of
                                plumes of ionizing radiation from illicit
                                nuclear reprocessing activities.
  \begin{list2}
    \item Technical report selected for publication in Journal of Undergraduate
            Science (1 of 18 winners out of 600 applicants).
    \item Winner of national SULI poster competition at AAAS conference.
  \end{list2}
\end{list1}

\vspace{20px}
\section{\sc Publications}

% Force references to appear even though they are not cited in the document
\nocitepublications{*}
\bibliographystylepublications{resume}
% Remove 'References' Heading
\renewcommand{\section}[2]{}
\bibliographypublications{bibliographies/publications}
% Recover sectioning 
\renewcommand{\section}{\resumesection}

\section{\sc Conference Presentations}

\nocitepresentations{*}
\bibliographystylepresentations{resume}
% Remove 'References' Heading
\renewcommand{\section}[2]{}
\bibliographypresentations{bibliographies/presentations}
% Recover sectioning 
\renewcommand{\section}{\resumesection}

%\section{\sc Posters}
%
%\nociteposters{*}
%\bibliographystyleposters{resume}
%% Remove 'References' Heading
%\renewcommand{\section}[2]{}
%\bibliographyposters{bibliographies/posters}
%% Recover sectioning 
%\renewcommand{\section}{\resumesection}

\section{\sc Teaching\footnotemark}
\footnotetext{All graduate student instructor (GSI) positions were full 20 
hr/week appointments}

%%%%%%%%%%%%%%%%%%%%%%%%%%%%%%%%%%%%%%%%%%%%%%%%%%%%%%%%%%%%%%%%%%%%%%%%%%%%%%%
{\bf Lecturer | UC Berkeley NE 204} \hfill Fall 2018

\begin{list1}
\item[] Graduate: 
        \href{http://ne204-fall2018.github.io}
             {Advanced Concepts in Radiation Detection and Measurement}
\end{list1}

%%%%%%%%%%%%%%%%%%%%%%%%%%%%%%%%%%%%%%%%%%%%%%%%%%%%%%%%%%%%%%%%%%%%%%%%%%%%%%%
{\bf Graduate Student Instructor | UC Berkeley STAT 159/259} \hfill Fall 2015 

\begin{list1}
\item[] Upper Division\footnotemark/Graduate: 
        \href{http://www.jarrodmillman.com/stat159-fall2015/}
             {Reproducible and Collaborative Statistical Data Science}
\end{list1}
\footnotetext{Upper Division = Juniors/Senior-level, Lower Division = 
Freshman/Sophomore-level}
%%%%%%%%%%%%%%%%%%%%%%%%%%%%%%%%%%%%%%%%%%%%%%%%%%%%%%%%%%%%%%%%%%%%%%%%%%%%%%%
{\bf Graduate Student Instructor | UC Berkeley NE 92} \hfill Fall 2014

\begin{list1}
  \item[] Lower Division: Introduction to Nuclear Engineering
\end{list1}
%%%%%%%%%%%%%%%%%%%%%%%%%%%%%%%%%%%%%%%%%%%%%%%%%%%%%%%%%%%%%%%%%%%%%%%%%%%%%%%
{\bf Graduate Student Instructor | UC Berkeley NE 120} \hfill Fall 2012

\begin{list1}
  \item[] Upper Division: Nuclear Materials
\end{list1}
%%%%%%%%%%%%%%%%%%%%%%%%%%%%%%%%%%%%%%%%%%%%%%%%%%%%%%%%%%%%%%%%%%%%%%%%%%%%%%%
{\bf Graduate Student Instructor | UC Berkeley NE 120} \hfill Fall 2011

\begin{list1}
  \item[] Upper Division: Nuclear Materials
\end{list1}
%%%%%%%%%%%%%%%%%%%%%%%%%%%%%%%%%%%%%%%%%%%%%%%%%%%%%%%%%%%%%%%%%%%%%%%%%%%%%%%
{\bf Graduate Student Instructor | UC Berkeley NE 104} \hfill Spring 2011

\begin{list1}
  \item[] Upper Division: Nuclear Instrumentation Laboratory
\end{list1}
%%%%%%%%%%%%%%%%%%%%%%%%%%%%%%%%%%%%%%%%%%%%%%%%%%%%%%%%%%%%%%%%%%%%%%%%%%%%%%%
{\bf Graduate Student Instructor | UC Berkeley NE 101} \hfill Fall 2010

\begin{list1}
  \item[] Upper Division: Introduction to Nuclear Physics
\end{list1}

\section{\sc Honors and Awards} 

Best Oral Presentation Award - University Industry Technology Interchange \hfill 2016 \\
Nuclear Science and Security Consortium Fellow \hfill 2014 - 2016 \\
American Nuclear Society Graduate Scholarship \hfill 2014 \\
UC Berkeley Outstanding GSI Award \hfill 2012 \\
University of Michigan Joseph M. Geisinger Scholar \hfill 2006 - 2010 \\
University of Michigan Class of 1931E Honors Scholar \hfill 2007 - 2010 \\
Sustainable Energy Fellowship Program Awardee, Cornell University \hfill 2010 \\
NEUP Scholarship Recipient \hfill 2009 - 2010 \\
American Nuclear Society National Scholarship Recipient \hfill 2007 - 2010 \\

\section{\sc Activities}

{\bf The Hacker Within, UC Berkeley Chapter} \hfill 2014-2016 \\ 
{\em Participant/Contributor} 

\begin{list2}
  \item Contributed presentations/breakout sessions on scientific computing
        with python, including machine learning (scikit-learn), data
        visualization, distributed computing, and extending python 
        (cython \& the Python C-API).
  \item \url{www.thehackerwithin.org/berkeley/previous.html}
\end{list2}

{\bf Nuclear Engineering Graduate Student Association | NEGSA} \hfill 2012-2015\\
{\em Co-founder \& Officer} 

\begin{list2}
  \item Officer position: Secretary/Treasurer \hfill 2012-2014
  \item Organized graduate student visit weekends and intra-departmental 
        educational and social events.
  \item Initiated computing seminars for discussion of advanced computing 
        topics relevant for nuclear engineers (merged with The Hacker Within,
        2014).
\end{list2}

{\bf American Nuclear Society (ANS)} \hfill 2007-2015 \\
{\em University of Michigan \& UC Berkeley Student Chapters} 

\begin{list1}
  \item[] Officer Positions
  \begin{list2}
    \item Social Chair (UCB) \hfill 2012-2013
    \item Vice President (UM) \hfill 2009-2010
    \item Student Conference Volunteer Coordinator (UM) \hfill 2010
    \item Outreach Chair (UM) \hfill 2008-2009
  \end{list2}
\end{list1}

%{\bf Tau Beta Pi} \hfill 2009-2010 \\
%{\em Member, MI-$\gamma$} 

%\section{\sc Computer Skills}
%
%% Consider adding section here

\end{resume}
\end{document}
